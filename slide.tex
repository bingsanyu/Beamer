\documentclass{beamer}
\usepackage{ctex, hyperref}
\usepackage[T1]{fontenc}
\usepackage{color}

% other packages
\usepackage{latexsym,amsmath,xcolor,multicol,booktabs,calligra}
\usepackage{graphicx,pstricks,listings,stackengine}
% 首页
\linespread{2}
\author{汇报人:胡兴发\ \ \ \\导\ \ \ 师:张本龚} %更改姓名
\title{本科生论坛交流分享} %更改标题
\subtitle{竞赛问题的简化方法} %更改副标题
\institute{数理科学学院}
\date{2022年9月29日} %更改时间
\institute[IaHS]{}
\usepackage{WTU}
\usepackage{setspace}
% defs
\def\cmd#1{\texttt{\color{red}\footnotesize $\backslash$#1}}
\def\env#1{\texttt{\color{blue}\footnotesize #1}}
\definecolor{deepblue}{rgb}{0,0,0.5}
\definecolor{deepred}{rgb}{0.6,0,0}
\definecolor{deepgreen}{rgb}{0,0.5,0}
\definecolor{halfgray}{gray}{0.55}

\lstset{
    basicstyle=\ttfamily\small,
    keywordstyle=\bfseries\color{deepblue},
    emphstyle=\ttfamily\color{deepred},    % Custom highlighting style
    stringstyle=\color{deepgreen},
    numbers=left,
    numberstyle=\small\color{halfgray},
    rulesepcolor=\color{red!20!green!20!blue!20},
    frame=shadowbox,
}
\usefonttheme[onlymath]{serif}
% 文档开始
\begin{document}
% 设置默认字体
\kaishu
% 标题页
\begin{frame}
    \titlepage
\end{frame}
% 标题页显示设置






\section{数学竞赛问题的简化方法}
\subsection{反例的构造}
\subsection{问题的简化}


\begin{frame}{反例构造1}
\begin{spacing}{2.0}
\textcolor{red}{例1.}若正项级数$\sum\limits_{n=0}^{\infty}u_n$收敛,则$u_n\to 0$,而$\int_{a}^{+\infty}f(x)dx $收敛一般不意味着$f(x)\to 0(x\to  +\infty)$.\\ \vspace{.5cm}例如$\int_{0}^{+\infty}\sin x^2dx =  \int_{0}^{+\infty} \frac{\sin t}{2\sqrt t}dt(x=\sqrt t)$收敛,但$\sin x^2\nrightarrow 0$(当$x\to +\infty$时).
	\end{spacing}
\end{frame}

\begin{frame}{反例构造2}
		\textcolor{red}{例2.}没有原函数的可积函数.\\
		设
		$$
		f\left( x \right) =\left\{ 
			\begin{array}{l}
			0,\ -1\le x\le 0\\
			1,\ \ \ 0\le x\le 1\\
			\end{array} \right. 
		$$
		
		
		$
		\text{易见,}
		f
		\text{在区间}\left[ -1,1 \right] 
		$
		上可积.\\然而,
		$
		f\text{在}\left[ -1,1 \right] 
		$
		上没有原函数.
\end{frame}
\begin{frame}{反例构造2}
	\begin{spacing}{1.5}
		
事实上,如果$\left[ -1,1 \right] \text{上有原函数}F\text{,即}$
$$F^{'}\left( x \right) =f\left( x \right) ,\ -1\le x\le 1,$$
那么,由达布$\left( Darboux \right)$定理可知,$F^{'}$应取得$F^{'}\left( -1 \right) =0$与\\
$F^{'}\left( 1 \right)=1$之间的每一个值,即$f$应该取得$f\left( -1 \right) =0$与
$f\left( 1 \right)$\\
$=1$之间的每一个值.此与$f$的定义相矛盾.因此,$f\text{在}\left[ -1,1 \right] $上\\
没有原函数.
\color{gray}{顺便指出,由$Darboux\text{定理可知,}$任何有跳跃间断点的函数都不可能有原函数.}

\end{spacing}
\end{frame}


\begin{frame}{反例构造3}
	\begin{spacing}{1.5}
		\textcolor{red}{例3.}积分的极限不等于极限的积分的函数列.\\
		在闭区间$\left[ 0,1 \right]$上如下定义函数列:
		$$
		f_n\left( x \right) =\left\{ \begin{array}{l}
			n,\,\,\,\,0<x\le \frac{1}{n},\\
			0,\,\,\,\,x=0\text{或}\frac{1}{n}<x\le 1.\\
		\end{array} \right. 
		$$
		易见,对每一正整数$n,f_n$在$\left[ 0,1 \right]$上都是可积的,且
		$$\lim_{n\rightarrow \infty}\int_0^1{f_n\left( x \right) dx=1.}$$
	\end{spacing}
\end{frame}


\begin{frame}{反例构造3}
	\begin{spacing}{1.5}
		但是,
		$$\int_0^1{\lim_{n\rightarrow \infty}}f_n\left( x \right) dx=\int_0^1{0dx=0.}$$
		因此,
		$$\lim_{n\rightarrow \infty}\int_0^1{f_n\left( x \right) dx\ne}\int_0^1{\lim_{n\rightarrow \infty}}f_n\left( x \right) dx.$$
		
	\end{spacing}
\end{frame}


\begin{frame}{反例构造3}
	\begin{spacing}{1.5}
		更为极端的例子是
		$$f_n\left( x \right) =\left\{ \begin{array}{l}
			2n^3x,\ \ \ \ \ \ 0\le x\le \frac{1}{2n}\\
			n^2-2n^3\left( x-\frac{1}{2n} \right) ,\ \frac{1}{2n}\le x\le \frac{1}{n}\\
			0,\ \frac{1}{n}\le x\le 1\\
		\end{array} \right. 
		$$
		此时,对任何$b\epsilon \left( 0,1 \right]$
		,都有
		$$\lim_{n\rightarrow \infty}\int_0^b{f_n\left( x \right) dx=}\lim_{n\rightarrow \infty}\frac{n}{2}=\infty ,$$
	\end{spacing}
\end{frame}


\begin{frame}{反例构造3}
	\begin{spacing}{1.5}
		而
		$$\int_0^b{\lim_{n\rightarrow \infty}f_n\left( x \right) dx=}\int_0^b{0dx=0.}$$
	\end{spacing}
\end{frame}

\begin{frame}{问题简化1}
	\begin{spacing}{1.5}
		\textcolor{red}{例4.}设$\lim\limits_{n \rightarrow + \infty}a_{n}=a, \lim\limits_{n \rightarrow + \infty}b_{n}=b$.证明:\par
$$\lim_{n \rightarrow + \infty}\frac{a_{1}b_{n}+a_{2}b_{n-1}+ \cdots +a_{n}b_{1}}{n}=ab$$
证:能否“不妨设”$b=0$?\par 记$\beta _{n}=b_{n}-b$,$\alpha  _{n}=a_{n}-a$,则$\lim\limits _{n \rightarrow + \infty}\beta _{n}=0$,
$\lim\limits _{n \rightarrow + \infty}\alpha _{n}=0$
\begin{flalign}
& \lim_{n \rightarrow + \infty}\frac{a_{1}b_{n}+a_{2}b_{n-1}+ \cdots +a_{n}b_{1}}{n}&
\nonumber\\
=& \lim_{n \rightarrow + \infty}\frac{a_{1}\beta _{n}+a_{2}\beta _{n-1}+ \cdots +a_{n}\beta _{1}}{n}
+b \lim _{n \rightarrow + \infty}\frac{a_{1}+a_{2}+ \cdots +a_{n}}{n}&\nonumber
\end{flalign}
\end{spacing}
\end{frame}

\begin{frame}{问题简化2}
	\begin{spacing}{1.2}
		\textcolor{red}{例5.}设$f\left( x \right) \text{在}\left( a,b \right) \text{内二阶可导,}c\text{为}\left( a,b \right) \text{内一点,满足}f^{''}\left( c \right) \ne 0.\text{则}
		$
		$
		\text{在}\left( a,b \right) \text{内存在}x_1\ne x_2\text{使得}
		$
		$$
		\frac{f\left( x_2 \right) -f\left( x_1 \right)}{x_2-x_1}=f^{'}\left( c \right) .
		$$
		证:能否“不妨设”$f^{'}\left( c \right) =0\text{?}$
		$
		\text{记}F\left( x \right) =f\left( x \right) -f^{'}\left( c \right) x.\text{则题设条件化为}
		$
		$$
		F^{'}\left( c \right) =0\text{,}F^{''}\left( c \right) =f^{''}\left( c \right) \ne 0.
		$$
		$
		\left( \text{这意味着}\boldsymbol{c}\text{是}F\left( x \right) \text{的严格极大值点或严格极小值点} \right) 
		$
	\end{spacing}
\end{frame}

\begin{frame}{问题简化2}
	\begin{spacing}{1.5}
		$\text{而需要求证的化为:寻找两个点有相同的函数值}.
		$
		$
		\text{证明:不妨设}F^{''}\left( c \right) >0,\text{且}c\text{是}F\left( x \right) \text{的严格极小值点}.
		$
		$
		\text{于是存在}\alpha ,\beta \text{有:}a<\alpha <c<\beta <b
		$
		,使得
		$F\left( \alpha \right) >F\left( c \right)
		$且
		$F\left( \beta \right) >F\left( c \right).
		$\\
		$
		\text{当}F\left( \alpha \right) =F\left( \beta \right) \text{时,问题已经得证}.
		$
		$
		\text{当}F\left( \alpha \right) >F\left( \beta \right) \text{时,因为}F\left( c \right) <F\left( \beta \right) <F\left( \alpha \right)
		$
		,由连续函数的介质定理可知:
		$
		\text{存在}\xi \epsilon \left( \alpha ,c \right) ,\text{使得}F\left( \xi \right) =F\left( \beta \right) \text{,结论成立。}
		$
		$
		\text{同理可证,当}F\left( \alpha \right) <F\left( \beta \right) \text{时结论也成立。}
		$
	\end{spacing}
\end{frame}

\begin{frame}{问题简化3}
	\begin{spacing}{1.0}
		$
		\text{证明:}\left( 1+\frac{1}{\boldsymbol{x}} \right) ^{\boldsymbol{x}}<\boldsymbol{e}<\left( 1+\frac{1}{\boldsymbol{x}} \right) ^{\boldsymbol{x}+1},\forall \boldsymbol{x}>0.
		$\\
		证:
	\begin{equation*}
		\begin{aligned}
		&\ \ \ \ \left( 1+\frac{1}{\boldsymbol{x}} \right) ^{\boldsymbol{x}}<\boldsymbol{e}<\left( 1+\frac{1}{\boldsymbol{x}} \right) ^{\boldsymbol{x}+1},&&\forall \boldsymbol{x}>0.\\
		&\Leftrightarrow \boldsymbol{x}\ln \left( 1+\frac{1}{\boldsymbol{x}} \right) <1<\left( \boldsymbol{x}+1 \right) \ln \left( 1+\frac{1}{\boldsymbol{x}} \right),&& \forall \boldsymbol{x}>0.\\
		&\Leftrightarrow \frac{1}{\boldsymbol{x}+1}<\ln \left( 1+\frac{1}{\boldsymbol{x}} \right) <\frac{1}{\boldsymbol{x}},&&\forall \boldsymbol{x}>0.\\
		&\Leftrightarrow 1-\frac{1}{\boldsymbol{x}+1}<\ln \left( 1+\frac{1}{\boldsymbol{x}} \right) <\boldsymbol{x}, &&\forall \boldsymbol{x}>0.\\
		&\Leftrightarrow \ln \left( 1+\boldsymbol{x} \right) <\boldsymbol{x}, &&\forall \boldsymbol{x}>-1,\boldsymbol{x}\ne 0.
		\end{aligned}
	\end{equation*}
	\end{spacing}
\end{frame}

\begin{frame}{思考题}
		求出使下列不等式对任意自然数$n$都成立的最大自然数$\alpha$和最小数$\beta \text{:}$
		$$
		\left( 1+\frac{1}{\boldsymbol{n}} \right) ^{\boldsymbol{n}+\boldsymbol{\alpha }}\le \boldsymbol{e}\le \left( 1+\frac{1}{\boldsymbol{n}} \right) ^{\boldsymbol{n}+\boldsymbol{\beta }.}
		$$
\end{frame}

\section{建模问题的简化方法——目标建模法}

\end{document}