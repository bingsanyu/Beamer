\documentclass{beamer}
\usepackage{ctex, hyperref}
\usepackage[T1]{fontenc}
\usepackage{color}

% other packages
\usepackage{latexsym,amsmath,xcolor,multicol,booktabs,calligra}
\usepackage{graphicx,pstricks,listings,stackengine}
% 首页
\author{汇报人:胡兴发} %更改姓名
\title{本科生论坛交流分享} %更改标题
\subtitle{竞赛问题的简化方法} %更改副标题
\institute{数理科学学院}
\date{2022年9月29日} %更改时间
\institute[IaHS]{指导老师:张本龚}
\usepackage{WTU}
\usepackage{setspace}
% defs
\def\cmd#1{\texttt{\color{red}\footnotesize $\backslash$#1}}
\def\env#1{\texttt{\color{blue}\footnotesize #1}}
\definecolor{deepblue}{rgb}{0,0,0.5}
\definecolor{deepred}{rgb}{0.6,0,0}
\definecolor{deepgreen}{rgb}{0,0.5,0}
\definecolor{halfgray}{gray}{0.55}

\lstset{
    basicstyle=\ttfamily\small,
    keywordstyle=\bfseries\color{deepblue},
    emphstyle=\ttfamily\color{deepred},    % Custom highlighting style
    stringstyle=\color{deepgreen},
    numbers=left,
    numberstyle=\small\color{halfgray},
    rulesepcolor=\color{red!20!green!20!blue!20},
    frame=shadowbox,
}
\usefonttheme[onlymath]{serif}
% 文档开始
\begin{document}
% 设置默认字体
\kaishu
% 标题页
\begin{frame}
    \titlepage
\end{frame}
% 标题页显示设置






\section{数学竞赛问题的简化方法}
\subsection{反例的构造}
\subsection{问题的简化}


\begin{frame}{反例构造1}
\begin{spacing}{2.0}
\textcolor{red}{例1.}若正项级数$\sum\limits_{n=0}^{\infty}u_n$收敛,则$u_n\to 0$,而$\int_{a}^{+\infty}f(x)dx $收敛一般不意味着$f(x)\to 0(x\to  +\infty)$.\\ \vspace{.5cm}例如$\int_{0}^{+\infty}\sin x^2dx =  \int_{0}^{+\infty} \frac{\sin t}{2\sqrt t}dt(x=\sqrt t)$收敛,但$\sin x^2\nrightarrow 0$(当$x\to +\infty$时).
	\end{spacing}
\end{frame}

\begin{frame}{反例构造2}
	
	\begin{spacing}{2.0}

		\textcolor{red}{例2.}没有原函数的可积函数.\\
		设
		
		$$
		f\left( x \right) =\left\{ 
			\begin{array}{l}
			0,\ -1\le x\le 0\\
			1,\ \ \ 0\le x\le 1\\
			\end{array} \right. 
		$$
		
		易见,
		$$
		f
		\text{在区间}\left[ -1,1 \right] 
		\text{上可积。然而,}f\text{在}\left[ -1,1 \right] 
		\text{上没有原函数。事实上,如果}f\text{在}
		$$

		$$
		\left[ -1,1 \right] \text{上有原函数}F\text{,即}F^{'}\left( x \right) =f\left( x \right) ,\ -1\le x\le 1
		$$
	
		
	\end{spacing}
\end{frame}




\begin{frame}{问题简化}
设$\lim\limits_{n \rightarrow + \infty}a_{n}=a, \lim\limits_{n \rightarrow + \infty}b_{n}=b$.证明:\par
$$\lim_{n \rightarrow + \infty}\frac{a_{1}b_{n}+a_{2}b_{n-1}+ \cdots +a_{n}b_{1}}{n}=ab$$
证:能否“不妨设”$b=0$?\par 记$\beta _{n}=b_{n}-b$,$\alpha  _{n}=a_{n}-a$,则$\lim\limits _{n \rightarrow + \infty}\beta _{n}=0$,
$\lim\limits _{n \rightarrow + \infty}\alpha _{n}=0$
\begin{flalign}
& \lim_{n \rightarrow + \infty}\frac{a_{1}b_{n}+a_{2}b_{n-1}+ \cdots +a_{n}b_{1}}{n}&
\nonumber\\
=& \lim_{n \rightarrow + \infty}\frac{a_{1}\beta _{n}+a_{2}\beta _{n-1}+ \cdots +a_{n}\beta _{1}}{n}
+b \lim _{n \rightarrow + \infty}\frac{a_{1}+a_{2}+ \cdots +a_{n}}{n}&\nonumber
\end{flalign}
\end{frame}


\section{建模问题的简化方法——目标建模法}

\end{document}